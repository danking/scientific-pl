\documentclass[10pt,letter]{article}

\usepackage{amsmath}
\usepackage{url}
\usepackage{amssymb}
% packages that allow mathematical formatting

\usepackage{graphicx}
% package that allows you to include graphics

\usepackage{setspace}
% package that allows you to change spacing

\usepackage{cite}
% for BibTeX

\usepackage{algpseudocode}
\usepackage{algorithmicx}
\usepackage{algorithm}

\usepackage[sc,osf]{mathpazo}
\linespread{1.05}         % Palatino needs more leading
\usepackage[T1]{fontenc}

\fontencoding{T1}
\fontfamily{ppl}
\fontseries{m}
\fontshape{n}
\fontsize{10}{13}
% Set font size. The first parameter is the font size to switch to; the second
% is the \baselineskip to use. The unit of both parameters defaults to pt. A
% rule of thumb is that the baselineskip should be 1.2 times the font size.
\selectfont

\usepackage{parskip}
% Use the style of having no indentation with a space between paragraphs.

\usepackage{enumitem}
% Resume an enumerated list, continuing the old numbering, after some
% intervening text.

% \usepackage{fullpage}
% package that specifies normal margins

\usepackage{titling}
% move the title up a bit, wastes less space
\setlength{\droptitle}{-5em}

\usepackage{caption}
\usepackage{subcaption}
% for subfigures

\usepackage{MnSymbol}
% for \llangle, i.e. double angle brackets

\DeclareSymbolFont{bbold}{U}{bbold}{m}{n}
\DeclareSymbolFontAlphabet{\mathbbold}{bbold}

%%%%%%%%%%%%%%%%%%%%%%%%%%%%%%%%%%%%%%%%%%%%%%%%%%%%%%%%%%%%%%%%%%%%%%%%%%%%%%%
%% Macros

\newcommand{\pipe}[0]{\;|\;}

\newcommand{\nats}[0]{\mathbbold{N}}
\newcommand{\rats}[0]{\mathbbold{Q}}

\renewcommand{\plus}[0]{+}
\newcommand{\funcArrow}[0]{\to}
\newcommand{\matchArrow}[0]{\Rightarrow}
\newcommand{\constraintArrow}[0]{\Rightarrow}

\newcommand{\unitType}[0]{\mathbbold{1}}
\newcommand{\nullType}[0]{\mathbbold{0}}

\newcommand{\functionApp}[4]{#1 [#2] [#3] (#4)}
\newcommand{\functionAppDotsp}[4]{\functionApp{#1}{#2_1\,#2_2\,\dots}{#3}{#4_1\,#4_2 \dots}}
\newcommand{\functionAppDotss}[4]{\functionApp{#1}{#2\,\dots}{#3}{#4\,\dots}}
\newcommand{\ctx}[0]{\mathcal{C}}

\newcommand{\functionValueDotss}[3]{\left\llangle #1\;\dots \;,\; #2\;\dots \;,\; #3 \right\rrangle}

\newcommand{\config}[3]{\left\langle #1 \;,\; #2 \;,\; #3 \right\rangle}

\newcommand{\envLocal}[0]{\Sigma_L}
\newcommand{\envGlobal}[0]{\Sigma_G}
\newcommand{\withMapping}[3]{#1\left[#2 \to #3 \right]}
\newcommand{\withMappings}[3]{\withMapping{#1}{#2}{#3}}

\newcommand{\stepsTo}[2]{#1 \longrightarrow #2}

\newcommand{\firstAlt}[2]{\;\;&#1 & \text{#2}\\}
\newcommand{\alt}[2]{| \firstAlt{#1}{#2}}

%%%%%%%%%%%%%%%%%%%%%%%%%%%%%%%%%%%%%%%%%%%%%%%%%%%%%%%%%%%%%%%%%%%%%%%%%%%%%%%

\begin{document}

\begin{align*}
  T ::= \firstAlt{\nats}{The naturals}
        \alt{T \times T}{Binary products}
        \alt{\unitType}{Unit type}
        \alt{T \plus T}{Sums}
        \alt{\nullType}{The empty type}
        \alt{array_n[T]}{n-dimensional arrays}
        \alt{T \funcArrow T}{Function types}
        \alt{\tau}{Type variables}
\end{align*}

\begin{align*}
  S ::= \firstAlt{T}{Flat types}
        \alt{C_1 \, \tau_1 \;\; C_2 \, \tau_2\; \dots \constraintArrow T}
            {Type with class constraint}
        \alt{\forall \,\tau_1\, \tau_2 \dots\; S}{Polymorphism}
\end{align*}

\begin{align*}
  e \in E ::= \firstAlt{x}{Variables}
        \alt{1 \;|\; 2 \;|\; \dots}{Natural numbers}
        \alt{E_1 \; \oplus \; E_2}{Primitive operations}
        \alt{\functionAppDotsp{f}{\tau}{I_1\,I_2\dots}{E}}
            {Function application}
        \alt{(E_1,\, E_2)}{Pairs}
        \alt{()}{Unit}
        \alt{\mathrm{inl}\; E \;|\; \mathrm{inr}\; E}{Sum construction}
        \alt{\mathrm{match}\; E \;\mathrm{with}\;
               P_1 \matchArrow E_1 \;|\;
               P_2 \matchArrow E_2 \dots
               \mathrm{end}}{Match}
        \alt{\{ \mathrm{for}\; x < E_1 : E_2 \} }{Array comprehension}
        \alt{E_1[E_2]}{Array subscript}
        \alt{E_1[E_2] := E_3}{Array update}
\end{align*}

\begin{align*}
  S ::= \firstAlt{E}{Expressions}
        \alt{S_1 \;;\; S_2}{Sequencing}
        \alt{\functionAppDotsp{f}{\tau}{C_1\,\tau_1\;C_2\,\tau_2\;\dots}{x}}
            {Function definition}
        \alt{\mathrm{class}\; \tau_{class}\; \tau \; \{ f_1 : \tau_1 \dots \} }
            {Type-class definition}
        \alt{\mathrm{instance}\; x \;:\; \tau_{class} \, \tau \{ f_1 = E_1 \dots \}}
            {Instance definition}
\end{align*}

\begin{align*}
  \ctx ::=& [] \\
     \pipe& \functionApp{\ctx}{\tau\;\dots}{I\;\dots}{e\;\dots} \\
     \pipe& \functionApp{v_f}{\tau\;\dots}{v_I\;\dots\;\ctx\;I\;\dots}{e\;\dots} \\
     \pipe& \functionApp{v}{\tau\;\dots}{v_I\;\dots}{v_E\;\dots\;\ctx\;e\;\dots} \\
     \vdots&
\end{align*}


\begin{align*}
  \stepsTo{&\config{\functionAppDotss{f}{\tau}{v_I\;\dots}{v}}
                  {\envLocal}
                  {\envGlobal^{\prime} =
                    \withMapping{\envGlobal}
                                {f}
                                {\functionValueDotss{I}{x}{e_{body}}}}\\ &}
          {\\&\config{e_{body}}
                  {\withMappings{
                   \withMappings{\envLocal}
                                {x\;\dots}
                                {v\;\dots}}
                                {I\;\dots}
                                {v_I\;\dots}
                              }
                  {\envGlobal^{\prime}}
                }
\end{align*}

\end{document}
